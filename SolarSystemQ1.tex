\section{Simulating the solar system}

In this section we look at question 1 of the fourth hand-in. 
For this question, I have made use of copied code from tutorial 12, where I have created the code in collaboration with Evelyn van der Kamp (s2138085). 
In particular, this is about the Leap Frog function. The Euler function is very similar to Leap Frog.

\subsection{Question 1a}

Code for 1(a):
\lstinputlisting[lastline=87]{NURHW4LizQ1.py}

For this part of the question I followed the example in the question to import the positions and velocities of the planets in the solar systems.
I converted the positions to AU, and plotted them in the $(x,y)$ plane and the $(x,z)$ plane.
The plots can be seen in Figure \ref{fig:SSxy} and Figure \ref{fig:SSxz}.

\begin{figure}[ht!]
  \centering
  \includegraphics[width=0.9\linewidth]{SolarSystemPresentxy.png}
  \caption{A plot of the x and y positions, in AU, of the planets in the solar system, at the current time, 2021-12-07 10:00.}
  \label{fig:SSxy}
\end{figure}

\begin{figure}[ht!]
  \centering
  \includegraphics[width=0.9\linewidth]{SolarSystemPresentxz.png}
  \caption{A plot of the x and z positions, in AU, of the planets in the solar system, at the current time, 2021-12-07 10:00.}
  \label{fig:SSxz}
\end{figure}

\subsection{Question 1b}

Code for 1(b) and (partly) 1(c):
\lstinputlisting[firstline=87, lastline=250]{NURHW4LizQ1.py}

In this part of the question, I created a function that calculates the acceleration due to the gravitational force, given by $\textbf{a} = -\frac{G*M_{\odot}*\textbf{r}}{r^3}$.
Here, the gravitational constant, $G$, and solar mass, $M_{\odot}$, are taken from astropy, and are in standard SI units, so the function assumes that the positions are in units of meters, and calculates the acceleration in m/s$^2$.
I subtracted the position of the sun from all other positions to make sure that the sun is in the origin, and stays there, assuming it is stationary.

I then imported the leap frog function from the tutorial, and slightly adjusted it to work for 3D positions and velocities.
I have also imported my very similar Euler function, which is just the leap frog function but then adjusted to apply Euler's method.
I created position arrays to save the integrated positions of all of the planets for both the leap frog and Euler's method and the integrated the positions and velocities over a time of 200 years, using a timestep of half a day.

This is done by first giving the velocity an initial kick based on the initial position and acceleration, to obtain the velocity half a step later.
This velocity is then used to calculate the position one step into the future.
Afterwards, the velocity is continued to be calculated a whole step after the first half step, and is used every time to calculate the position half a step later than the velocity.

The plots showing the results for the leap frog method can be seen in Figure \ref{fig:LFxy} and \ref{fig:LFtz}.

\begin{figure}[ht!]
  \centering
  \includegraphics[width=0.9\linewidth]{SolarSystemLeapFrogxy.png}
  \caption{A plot of the x and y positions, in AU, of the planets in the solar system, at the current time, 2021-12-07 10:00 until 200 years into the future. The positions and velocities have been integrated using the leap frog method.}
  \label{fig:LFxy}
\end{figure}

\begin{figure}[ht!]
  \centering
  \includegraphics[width=0.9\linewidth]{SolarSystemLeapFrogtz.png}
  \caption{A plot of the time and z positions, in AU, of the planets in the solar system, at the current time, 2021-12-07 10:00 until 200 years into the future. The positions and velocities have been integrated using the leap frog method.}
  \label{fig:LFtz}
\end{figure}

Leap frog is a suitable algorithm to use for physical problems like the orbits of planets, since this method is time reversible, and thus conserves energy.
Using a forward method would add energy into the system, while a backward method would take energy from the system. 

\subsection{Question 1c}

Code for the rest of 1(c):
\lstinputlisting[firstline=250]{NURHW4LizQ1.py}

In this part I have chosen to use Euler's method to do the same as in part 1(b).
I have chosen Euler's method, because it is similar to the leap frog method, but it gives a very different outcome.
This is because of what was mentioned before about leap frog conserving the energy, while a forward method like Euler's method adds energy into the system, and thus causes the planets to diverge from their orbits.

The divergence is especially clear in the orbit of Mercury, which ends up near the orbit of Uranus, because of the short period of Mercury, and thus 200 years is more than 200 orbits for Mercury, while it is only a handful of orbits for the outer planets, and only a little more than 1 orbit for Neptune, so the divergence is less clear the longer the period of the planet.

The same plots as for 1(b) can be seen in Figure \ref{fig:Euxy} and \ref{fig:Eutz}, while a comparison of the x positions of the leap frog and Euler's method can be seen in Figure \ref{fig:LFEdiff}.

\begin{figure}[ht!]
  \centering
  \includegraphics[width=0.9\linewidth]{SolarSystemEulerxy.png}
  \caption{A plot of the x and y positions, in AU, of the planets in the solar system, at the current time, 2021-12-07 10:00 until 200 years into the future. The positions and velocities have been integrated using Euler's method.}
  \label{fig:Euxy}
\end{figure}

\begin{figure}[ht!]
  \centering
  \includegraphics[width=0.9\linewidth]{SolarSystemEulertz.png}
  \caption{A plot of the time and z positions, in AU, of the planets in the solar system, at the current time, 2021-12-07 10:00 until 200 years into the future. The positions and velocities have been integrated using Euler's method.}
  \label{fig:Euxz}
\end{figure}

\begin{figure}[ht!]
  \centering
  \includegraphics[width=0.9\linewidth]{SolarSystemEulerLeapFrogDiff.png}
  \caption{A plot of the time versus the difference in x positions, in AU, of the planets in the solar system, at the current time, 2021-12-07 10:00 until 200 years into the future, comparing the leap frog and Euler's method.}
  \label{fig:LFEdiff}
\end{figure}






