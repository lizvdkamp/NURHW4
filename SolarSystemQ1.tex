\section{Simulating the solar system}

In this section we look at question 1 of the fourth hand-in. 
For this question, I have made use of copied code from tutorial 12, where I have created the code in collaboration with Evelyn van der Kamp (s2138085). 
In particular, this is about the Leap Frog function. The Euler function is very similar to Leap Frog.

\subsection{Question 1a}

Code for 1(a):
\lstinputlisting[lastline=87]{NURHW4LizQ1.py}

For this part of the question I followed the example in the question to import the positions and velocities of the planets in the solar systems.
I converted the positions to AU, and plotted them in the $(x,y)$ plane and the $(x,z)$ plane.
The plots can be seen in Figure \ref{fig:SSxy} and Figure \ref{fig:SSxz}.

\begin{figure}[ht!]
  \centering
  \includegraphics[width=0.9\linewidth]{SolarSystemPresentxy.png}
  \caption{A plot of the x and y positions, in AU, of the planets in the solar system, at the current time, 2021-12-07 10:00.}
  \label{fig:SSxy}
\end{figure}

\begin{figure}[ht!]
  \centering
  \includegraphics[width=0.9\linewidth]{SolarSystemPresentxz.png}
  \caption{A plot of the x and z positions, in AU, of the planets in the solar system, at the current time, 2021-12-07 10:00.}
  \label{fig:SSxz}
\end{figure}




